\subsection{Confusion matrix}
\begin{wrapfigure}{l}{0.6\textwidth}
	\includegraphics[width=\textwidth]{confusion_matrix.png}
	\caption{Confusion matrix of 10-fold CV predictions and true labels, white is 100\% accuracy, black is 0\%}
	\label{fig:confusion}
\end{wrapfigure}
The confusion matrix in figure \ref{fig:confusion} is built from MFCC+HCDF Fisher Vector feature set with KNN5 classifier. We see that Classical, Metal , Blues and Jazz are the best identified genres, with accuracy exceeding 70\%, this corroborates with our visualization in figure \ref{fig:tsne1} which showed clear clustering for those genres. Genre pairs with the most confusion are \{Reggae, Hip-Hop\}, \{Rock, Disco\}, \{Rock, Country\}, which also corroborates with our visualizations in figure \ref{fig:tsne2}. 
\subsection{Classifiers and features}
It is no surprise that the best single classifier before ensemble learning is linear SVM, because our underlying features are represented in Fisher Vectors, which lends itself to efficient linear classification.
The Fisher Kernel inherits advantages from both generative and discriminative models by building a kernel from a generative model (in this case GMM),  it characterizes a sample by its deviation from the model measured by computing the gradient of the sample log-likelihood with  respect  to  the  model  parameters.\cite{HAL}.

\subsection{Summary and Extensions}
Through our trials varying features and classifiers, we saw a maximum accuracy of 75.5\%. We found that genres of music that humans themselves may have difficulty distinguishing (for example, rock and country) similarly confused the classifiers. We also noticed that Fisher Vectors perform well with linear classifiers. 

A possible extension could be first merging the commonly confused classes and train hierarchical binary classifiers. Alternatively, we could train classifiers that return multiple labels if exceeding some confidence threshold.